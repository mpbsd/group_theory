\section{Group Actions}\label{sec:group-actions}

\begin{definition}\label{def:group-action}
  A right action of a group \(G\) on a nonempty set \(X\) is a function
  \[
    X\times{G}\to{X},\quad(x,g)\mapsto{xg},
  \]
  such that:
  \begin{enumerate}[i.]
    \item
      \(x(gh)=(xg)h\) for all \(g,h\in{G}\) and \(x\in{X}\);
    \item
      \(x1=x\) for all \(x\in{X}\).
  \end{enumerate}
  The set \(X\) is called a \(G\)-set. A left action is defined in a similar
  fashion.
\end{definition}

\begin{example}\label{ex:action-of-symmetric-groups}
  Let \(S_{n}\) be the symmetric group of degree \(n\). Then, \(S_{n}\) acts on
  the the set \(\left\{1,\ldots,n\right\}\) from the right in a rather natural
  way:
  \[
    \left\{1,\ldots,n\right\}\times{S_{n}}\to\left\{1,\ldots,n\right\},
    \quad
    (x,\alpha)\mapsto{x^{\alpha}}.
  \]
\end{example}

\begin{example}\label{ex:conjugation}
  Let \(G\) be a group. Then, \(G\) acts on itself from the right by conjugation:
  \[
    G\times{G}\to{G},\quad(x,g)\mapsto{x^{g}=g^{-1}xg}.
  \]
\end{example}

\begin{definition}\label{def:orbits-and-stabilizers}
  Let \(X\) be a \(G\)-set. Then, following common terminology, the set:
  \begin{enumerate}[i.]
    \item
      \(\text{orb}(x,G)=\left\{y\in{G}:y=gx\text{ for some }g\in{G}\right\}\) is called the
      \(G\)-orbit of \(X\);
    \item
      \(\text{stab}(x,G)=\left\{g\in{G}:xg=x\right\}\) is called the \(G\)-stabilizer of
      \(x\),
  \end{enumerate}
  for any \(x\in{X}\).
\end{definition}

\begin{proposition}\label{prop:orbit-decomposition-of-g-sets}
  Let \(X\) be a \(G\)-set. Then, the binary relation given by
  \begin{equation}\label{eq:g-equivalent-points}
    \forall{x,y\in{X}}:
    \quad
    x\equiv{y}\mod{G}
    \iff
    \exists{g\in{G}}:
    \,\,
    xg=y,
  \end{equation}
  is an equivalence relation on \(X\). Moreover, the equivalence class
  \[
    \left\{y\in{X}:x\equiv{y}\mod{G}\right\},
  \]
  equals \(\text{orb}(x,G)\), the \(G\)-orbit of \(x\), for any point \(x\in{X}\).
\end{proposition}

\begin{proof}
  For any given \(x,y\) and \(z\) in \(X\), we have that:
  \begin{enumerate}
    \item
      \(x\equiv{x}\mod{G}\) for every \(x\in{X}\), since \(x1=x\);
    \item
      If \(x\equiv{y}\mod{G}\), then \(xg=y\) for some \(g\in{G}\). But, then
      \(yg^{-1}=x\) and so \(y\equiv{x}\mod{G}\);
    \item
      If \(x\equiv{y}\mod{G}\) and \(y\equiv{z}\mod{G}\), then we have that
      \(xg=y\) and \(yh=z\) for certain \(g,h\in{G}\). Therefore,
      \(x(gh)=(xg)h=yh=z\) and so \(x\equiv{z}\mod{G}\).
  \end{enumerate}
  Now, notice that if \(y\in\left\{y\in{X}:x\equiv{y}\mod{G}\right\}\), then
  \(y=gx\) for some \(g\in{G}\). Conversely, for any \(g\in{G}\),
  \(gx\equiv{x}\mod{G}\) because \(g^{-1}\in{G}\) and
  \(g^{-1}(gx)=(g^{-1}g)x=1x=x\). Therefore, we conclude that
  \begin{align*}
    \left\{y\in{X}:x\equiv{y}\mod{G}\right\}
    &=
    \left\{gx:g\in{G}\right\}
    =\text{orb}(x,G).
  \end{align*}
  This completes the proof.
\end{proof}

Suppose that \(X\) is a finite \(G\)-set. Let \(T\subset{X}\) be a set with the
following properties:
\begin{enumerate}
  \item
    \(X=\bigcup\left\{\text{orb}(x,G):x\in{T}\right\}\).
  \item
    \(\forall{x,x'\in{T}}:\quad{x\neq{x'}}\implies{\text{orb}(x,G)\cap{\text{orb}(x',G)}}=\emptyset\);
\end{enumerate}
Then, it's clear that
\begin{equation}\label{eq:numeric-characteristic}
  \order{X}=\sum\limits_{x\in{T}}\order{\text{orb}(x,G)}=\sum\limits_{x\in{T}}(G:\text{stab}(x,G)).
\end{equation}

\begin{proposition}\label{prop:cardinality-of-orbits}
  Let \(X\) be a \(G\)-set. Then, for any \(x\in{X}\), \(\text{stab}(x,G)\) is a subgroup
  of \(G\) and the cardinality of \(\text{orb}(x,G)\), the \(G\)-orbit of \(x\), equals the
  index \((G:\text{stab}(x,G))\) of \(\text{stab}(x,G)\) in \(G\).
\end{proposition}

\begin{proof}
  Let \(x\in{X}\) be given. The identity element of \(G\) obviously belongs to
  \(\text{stab}(x,G)\) and, for any pair of elements \(g,h\in{\text{stab}(x,G)}\), we have that
  \[
    x(gh^{-1})=(xg)h^{-1}=xh^{-1}=(xh)h^{-1}=x(hh^{-1})=x1=x,
  \]
  and as such \(gh^{-1}\in{\text{stab}(x,G)}\). Therefore, \(\text{stab}(x,G)\) is a subgroup of
  \(G\). Now, regarding the function
  \[
    G/\text{stab}(x,G)\to{\text{orb}(x,G)},
    \quad
    \text{stab}(x,G)g\mapsto{xg}.
  \]
  it's true that
  \begin{align*}
    xg=xh
    &\iff
    x(gh^{-1})=x
    \iff
    gh^{-1}\in{\text{stab}(x,G)}
    \\
    &\iff
    \text{stab}(x,G)g=\text{stab}(x,G)h,
  \end{align*}
  for every pair of elements \(g,h\in{G}\), from what it follows that
  \(\text{stab}(x,G)g\mapsto{xg}\) is an injective function, as well as
  \[
    y\in{\text{orb}(x,G)}
    \iff
    \exists{g\in{G}}:\,
    y=xg
    \implies
    \text{stab}(x,G)g\mapsto{y=xg},
  \]
  which shows that \(\text{stab}(x,G)g\mapsto{xg}\) is also onto. Henceforth,
  \(\order{\text{orb}(x,G)}=(G:\text{stab}(x,G))\) as claimed. This completes the proof.
\end{proof}

\section{Applications}\label{sec:applications}

\begin{proposition}\label{prop:class-equation}
  Let \(G\) be a finite \(p\)-group. Then, \(G\) has a nontrivial center.
\end{proposition}

\begin{proof}
  Let \(G\) act on itself from the right by conjugation. Then, we have that
  \[
    \text{orb}(x,G)
    =
    \left\{x^{g}:g\in{G}\right\}
    =
    \left\{x\right\}
    \iff
    x\in{Z(G)},
  \]
  for any \(x\in{G}\). Therefore,
  \[
    \order{G}
    =
    \sum_{x\in{Z(G)}}\order{\text{orb}(x,G)}
    +
    \sum_{x\notin{Z(G)}}\order{\text{orb}(x,G)}
    =
    \order{Z(G)}
    +
    \sum_{x\notin{Z(G)}}(G:\text{stab}(x,G))
  \]
\end{proof}

\begin{theorem}[Cauchy]\label{thm:cauchys-theorem}
  Let \(G\) be a finite group and \(p\) be a prime divisor of \(\order{G}\).
  Then, there is some \(g\in{G}\) such that \(\order{g}=p\).
\end{theorem}

\begin{proof}
  The graph of the function
  \[
    f:G^{p-1}\to{G},\quad
    \left(x_{1},\ldots,x_{p-1}\right)\mapsto\left(\prod_{i=1}^{p-1}x_{i}\right)^{-1},
  \]
  is the set
  \[
    \Omega
    =
    \left\{(x_{1},\ldots,x_{p})\in{G^{p}}:\prod_{i=1}^{p}x_{i}=1\right\},
  \]
  which has \(\order{G}^{p-1}\) elements in total, a number divisible by \(p\).
  Consider the action of the additive group \(\mathbb{Z}_{p}\) on the set
  \(\Omega\) from the right given by
  \[
    \left(x_{1},x_{2}\ldots,x_{p-1},x_{p}\right)\cdot\bar{1}
    =
    \left(x_{p},x_{1}\ldots,x_{p-2},x_{p-1}\right).
  \]
  The \(\mathbb{Z}_{p}\)-orbit of a point
  \(x=\left(x_{1},\ldots,x_{p}\right)\in{\Omega}\) consists of the element
  \(x\) alone if, and only if, the coordinates
  \(x_{1},x_{2},\ldots,x_{p-1},x_{p}\) of \(x\) are all equal to one another,
  that is, \(x_{1}=x_{2}=\cdots=x_{p-1}=x_{p}\). This is certainly the case for
  the element \(\left(1,\ldots,1\right)\in{\Omega}\) whose coordinates are all
  equal to the identity element of \(G\). Let \(T\subset{\Omega}\) be a
  transveral for the action of \(\mathbb{Z}_{p}\) on \(\Omega\), meaning that:
  \begin{enumerate}
    \item
      \(\Omega=\bigcup\left\{\text{orb}(x,\mathbb{Z}_{p}):x\in{T}\right\}\);
    \item
      \(\forall{x,x'\in{T}}:\quad{x\neq{x'}\implies{\text{orb}(x,\mathbb{Z}_{p})\cap{\text{orb}(x',\mathbb{Z}_{p})}=\emptyset}}\).
  \end{enumerate}
  Then, we have that
  \[
    \order{\Omega}
    =
    \sum\limits_{x\in{T}}\order{\text{orb}(x,\mathbb{Z}_{p})}
    =
    \sum\limits_{\order{\text{orb}(x,\mathbb{Z}_{p})}=1}1
    +
    \sum\limits_{\order{\text{orb}(x,\mathbb{Z}_{p})}>1}(\mathbb{Z}_{p}:\text{stab}(x,\mathbb{Z}_{p})).
  \]
  Since
  \[
    \order{\Omega}
    \quad\text{and}\quad
    \sum_{\order{\text{orb}(x,\mathbb{Z}_{p})}=1}(\mathbb{Z}_{p}:\text{stab}(x,\mathbb{Z}_{p})),
  \]
  are both divisible by \(p\), so is
  \[
    \sum_{\order{\text{orb}(x,\mathbb{Z}_{p})}>1}1.
  \]
  This last sum would be equal to zero if there were no
  \(\mathbb{Z}_{p}\)-orbits of size \(1\) at all in \(\Omega\), but as we've
  already seen there's that of the element \(x=(1,\ldots,1)\). Therefore, there
  must exist some \(g\in{G}\), \(g\neq{1}\), with
  \[
    \text{orb}((x,\ldots,x),\mathbb{Z}_{p})=\left\{(x,\ldots,x)\right\},
  \]
  from what we get that \(x^{p}=1\). This completes the proof.
\end{proof}
