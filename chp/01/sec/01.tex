\begin{definition}\label{def:group-action}
  A right action of a group \(G\) on a nonempty set \(X\) is a function
  \[
    X\times{G}\to{X},\quad(x,g)\mapsto{xg},
  \]
  such that:
  \begin{enumerate}[i.]
    \item
      \(x(gh)=(xg)h\) for all \(g,h\in{G}\) and \(x\in{X}\);
    \item
      \(x1=x\) for all \(x\in{X}\).
  \end{enumerate}
  The set \(X\) is called a \(G\)-set. A left action is defined in a similar
  fashion.
\end{definition}

\begin{example}\label{ex:action-of-symmetric-groups}
  Let \(S_{n}\) be the symmetric group of degree \(n\). Then, \(S_{n}\) acts on
  the the set \(\left\{1,\ldots,n\right\}\) from the right in a rather natural
  way:
  \[
    \left\{1,\ldots,n\right\}\times{S_{n}}\to\left\{1,\ldots,n\right\},
    \quad
    (x,\alpha)\mapsto{x^{\alpha}}.
  \]
\end{example}

\begin{example}\label{ex:conjugation}
  Let \(G\) be a group. Then, \(G\) acts on itself from the right by conjugation:
  \[
    G\times{G}\to{G},\quad(x,g)\mapsto{x^{g}=g^{-1}xg}.
  \]
\end{example}

\begin{definition}\label{def:orbits-and-stabilizers}
  Let \(X\) be a \(G\)-set. Then, following common terminology, the set:
  \begin{enumerate}[i.]
    \item
      \(xG=\left\{y\in{G}:y=gx\text{ for some }g\in{G}\right\}\) is called the
      \(G\)-orbit of \(X\);
    \item
      \(G_{x}=\left\{g\in{G}:xg=x\right\}\) is called the \(G\)-stabilizer of
      \(x\),
  \end{enumerate}
  for any \(x\in{X}\).
\end{definition}

\begin{proposition}\label{prop:orbit-decomposition-of-g-sets}
  Let \(X\) be a \(G\)-set. Then, the binary relation given by
  \begin{equation}\label{eq:g-equivalent-points}
    \forall{x,y\in{X}}:
    \quad
    x\equiv{y}\pmod{G}
    \iff
    \exists{g\in{G}}:
    \,\,
    xg=y,
  \end{equation}
  is an equivalence relation on \(X\). Moreover, the equivalence class
  \[
    \left\{y\in{X}:x\equiv{y}\pmod{G}\right\},
  \]
  equals \(xG\), that is, the \(G\)-orbit of \(x\), for any point \(x\in{X}\).
\end{proposition}

\begin{proof}
  For any given \(x,y\) and \(z\) in \(X\), we have that:
  \begin{enumerate}
    \item
      \(x\equiv{x}\pmod{G}\) for every \(x\in{X}\), since \(x1=x\);
    \item
      If \(x\equiv{y}\pmod{G}\), then \(xg=y\) for some \(g\in{G}\). But, then
      \(yg^{-1}=x\) and so \(y\equiv{x}\pmod{G}\);
    \item
      If \(x\equiv{y}\pmod{G}\) and \(y\equiv{z}\pmod{G}\), then we have that
      \(xg=y\) and \(yh=z\) for certain \(g,h\in{G}\). Therefore,
      \(x(gh)=(xg)h=yh=z\) and so \(x\equiv{z}\pmod{G}\).
  \end{enumerate}
  Now, notice that if \(y\in\left\{y\in{X}:x\equiv{y}\pmod{G}\right\}\), then
  \(y=gx\) for some \(g\in{G}\). Conversely, for any \(g\in{G}\),
  \(gx\equiv{x}\pmod{G}\) because \(g^{-1}\in{G}\) and
  \(g^{-1}(gx)=(g^{-1}g)x=1x=x\). Therefore, we conclude that
  \begin{align*}
    \left\{y\in{X}:x\equiv{y}\pmod{G}\right\}
    &=
    \left\{gx:g\in{G}\right\}
    =xG.
  \end{align*}
\end{proof}

\begin{proposition}\label{prop:cardinality-of-orbits}
  Let \(X\) be a \(G\)-set. Then, we have:
  \begin{enumerate}[i.]
    \item
      \(G_{x}\leqslant{G}\);
    \item
      \(\order{xG}=(G:G_{x})\),
  \end{enumerate}
  for every \(x\in{X}\).
\end{proposition}

\begin{proof}
  Let \(x\in{X}\) be given. The identity element of \(G\) obviously belongs to
  \(G_{x}\) and, for any pair of elements \(g,h\in{G_{x}}\), we have that
  \[
    x(gh^{-1})=(xg)h^{-1}=xh^{-1}=(xh)h^{-1}=x(hh^{-1})=x1=x,
  \]
  and as such \(gh^{-1}\in{G_{x}}\). Therefore, \(G_{x}\) is a subgroup of
  \(G\). Now, regarding the function
  \[
    G/G_{x}\to{xG},
    \quad
    G_{x}g\mapsto{xg}.
  \]
  it's true that
  \[
    xg=xh
    \iff
    x(gh^{-1})=x
    \iff
    gh^{-1}\in{G_{x}}
    \iff
    G_{x}g=G_{x}h,
  \]
  for every pair of elements \(g,h\in{G}\), from what it follows that
  \(G_{x}g\mapsto{xg}\) is an injective function, as well as
  \[
    y\in{xG}
    \iff
    \exists{g\in{G}}:\,
    y=xg
    \implies
    G_{x}g\mapsto{y=xg},
  \]
  which shows that \(G_{x}g\mapsto{xg}\) is also onto. Henceforth,
  \(\order{xG}=(G:G_{x})\) as claimed. This completes the proof.
\end{proof}

